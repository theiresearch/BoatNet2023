\subsection{Energy Crisis, Energy Security and Climate Change}

The Intergovernmental Panel on Climate Change (IPCC) explains, in its latest report, that humans and nature are being pushed beyond their abilities to adapt due to the anthropogenic emission caused by economic, industrial and societal activities~\cite{IPCC2022Portner}. Nowadays, carbon-intensive resources still consist of a large proportion of the energy system~\cite{IPCC2022Portner} – about 80\% in 2017~\cite{raturi2019renewables}. However, the share of electricity production from renewables increased from 20.83\% to 28.98\% from 1985 to 2020~\cite{BP2021bp}. Still, carbon emissions have not been reducing in line with the ambitions of the Paris Agreement, and it is predicted that in the next few years, the gains in carbon reduction due to the COVID-19 pandemic will be erased faster than expected~\cite{WEO2021IEA}. However, even under all these pressures and projections, it is still possible for humanity to keep the global temperature below 1.5°C by 2100 if substantial changes are made to the current energy systems.

Energy security is an important part of the countries’ strategies to support economic growth and provide basic services to their populations. Currently, nations deposit most of their energy security onto fossil fuels while expanding their renewable power capacity. Fossil fuels and their conversion systems (e.g. internal combustion engines) permit operators to react quickly to changes in the energy demand (i.e. more control over the energy deployment) while offering acceptable volumetric energy densities. However, countries’ heavy reliance on fossil fuels coupled with the fuel’s geographical origin is at the mercy of important price fluctuations due to geopolitical and logistical events, such as Russia’s invasion of Ukraine. These can disrupt energy systems and vulnerate a country’s stability and human livelihoods~\cite{bhattacharyya2009fossil,russiagas}. On the other hand, renewable energy production and distribution tend to be within the country's boundaries. During the last few years, its price has been catching those of the subsidized fossil fuels – with some specific examples already undercutting fossil fuel prices~\cite{IRENA2021Renewable}. In fact, from 1987 to 2015, the cost of oil and coal rose by approximately 36\% and 81\% respectively and from 1989 to 2015, the cost of natural gas rose by approximately 53\%~\cite{BP2016bp}. Lately in the UK, the natural gas price during March 2022 has increased to around £5.40/therm, a rise above 1,100\% from the price levels seen in 2021~\cite{T2022UKNatureGas}. Still, it is important to mention that renewable energy variability and investment requirements are important challenges towards grid stability and energy security.

\subsection{Shipping Sector, Small-Boat Fleet and Emission Inventory}
Shipping, being the backbone of the market globalization, play an important role in the carbon reduction of human activities since it moves around 90\% of the goods around the globe~\cite{walker2019environmental}. Shipping reliance on fossil fuels coupled with a strong economic growth saw shipping total \ch{CO2} emissions grow from 962 Mt in 2012 to 1,056 Mt in 2018, which represented about 2.9\% of the total global emissions ~\cite{IMO2021Fourth}. Further, if nothing is done in the sector, it was projected that by 2050 shipping \ch{CO2} emissions could grow up to 1,500 Mt. In this light, the International Maritime Organization (IMO) produced in 2018 its ambitions to decarbonize the international shipping~\cite{imo2018adoption}. However, this vision only covers international navigation composed of large vessels and missing the small boat fleet – vessels below 100 gross tonnage which tend to measure less than 24 m in length~\cite{uk2021Operational}.

But there are good reasons for that, from the IMO perspective its focus is mainly on ships that navigate on international waters or large ships performing domestic voyages. These vessels are required to have AIS transponders for safety navigation. Small boats tend not to have an AIS transponder which makes it difficult to study their movements. Small boats are register and monitored by the national and regional bodies and the comprehensiveness of it depends on the capital and human resources plus the infrastructure to maintain the registry. As well, small boats are a diverse segment of shipping which built and usage depends on the geographical location, type of activity, cost of building and operating, and accessibility to fuel or bunkering infrastructure. Similarly, engine providers are extensive, giving a broad range of fuel consumption curves and emissions. Furthermore, the fuel selection is equally diverse: petrol, diesel, petrol mixed with engine oil – mainly for 2-stroke engines, ethanol and bio-fuels – or a mix of bio-fuel with fossil fuels. Finally, on this matter, not all of the small boats are powered by an internal combustion engine, they can be powered by sail, battery electric or paddles.

Still, under all these challenges the small boat fleet is highly important to estimate this shipping segment emission footprint based on its activity. Emissions inventories can help understand what measures need to be taken to enable the industry to start the road to full decarbonization in a just and equitable way. Although it is possible to calculate the emissions from large vessels from the satellite data sent from the ship's transponder and coupled to technical models~\cite{IMO2021Fourth}, the small vessels depend on the national registration system, and their operation is normally assumed or captured by national fuel sales which tend to be highly aggregated. Developed economies such as the UK tend to have a national registry of the smaller vessels~\cite{uk2021registration} that allows to have a sense of the level activity and hence infer the \ch{CO2} emissions.

However, in developing countries, it tends to be a mixed bag on the level of precision and availability. For instance, in Mexico, only fishing vessels are counted into the registry ~\cite{Mexico2021RegisteredVessels}. Still, it is not easy to know where they are located and infer their activities. Besides, the rest of the small-boat categories are not registered. In all, Mexico does not have a regional \ch{CO2} inventory specialized on the small-boat fleet, instead they are aggregated as part of the \textit{maritime and fluvial navigation [1A3d]} class in the national annual emission inventory developed by the Instituto Nacional de Ecología y Cambio Climático (INECC)~\cite{inecc2020inventario} in a top-down approach recommended by the IPCC~\cite{eggleston20062006}. Therefore, quantifying and categorizing the small-boat fleet will allow a better precision of where and how the emissions are being emitted and will give a better understanding the maritime emission inventory.

Observing the shipping activity in the Gulf of California is essential due to its unique geographical location, conformation, and biophysical environment~\cite{LLUCHCOTA20071}~\cite{munguia2018ecological}~\cite{MARINONE2012133}. In the Gulf of California, there is the largest fish producing state (Sonora) in Mexico~\cite{MELTZER2006222} and the most prominent sports fishing destination (Los Cabos, Baja)~\cite{hernandez2012economic}. Besides, it is one of the regions in Mexico with important protected areas due to the flora and fauna diversity such as the upper part of the Gulf of California, Bahia Loreto and Bahia los Angeles~\cite{CNANP2022Atlas, SMARN2022Islas}. 



\subsection{Bringing Deep Learning to Small Ships Detection in Satellite Images}
Bringing deep learning, especially convolutional neural networks, to the field of satellite image recognition is essential. First, the technologies barriers that satellite image recognition required is gradually decreasing with time. As the quality and quantity of global satellite images improve, obtaining the same or even better detection results with reduced parameters and complexity of convolutional neural networks is possible. Second, satellite image recognition is not new. Scholars at the University of Texas at Austin have been using machine learning and remote sensing imagery to discover undersea shipwrecks successfully~\cite{character2021archaeologic}. However, discovering submarine wrecks does not require algorithms to describe the location and size of the wreck very precisely, as opposed to identifying small boats and their physical characteristics. Therefore, in this study, accurately identifying and measuring the dimensions of small ships is a current challenge and the main objective of this work.


\subsection{Contributions}
The contributions of this study are summarized as follows:
\begin{itemize}
    \item We use the best algorithm, which we call BoatNet. This work shows that BoatNet detects many small boats in low-resolution, blurry satellite images with considerable noise levels. The precision of training can be up to 96\% and the precision of testing the data set of the Gulf of California can be up to \textbf{...\%}).

    \item This work demonstrated that BoatNet can detect the length of small boats with a precision up to \textbf{...\%}). 

    \item We created a methodology that can detect small boats and that has allowed us to have a better understanding of their activity and physical characteristics. Based on this, we answered questions about the composition of small boats in the Gulf of California. We believe this is the first but important step in constructing a way to estimate maritime carbon footprint.
\end{itemize}