
This study demonstrated the capabilities of a deep learning approach for the automatic detection and identification of small boats in the waters surrounding three cities in the Gulf of California with a precision of up to 74.0\%. This work used CNNs to identify types of small vessels. Specifically, this study presented an image detection model, BoatNet, capable of distinguishing small boats in the Gulf of California with an accuracy of up to 93.9\% and encouraging results considering the high variability of the input images.

Even with the model level of performance using large and highly ambiguous training images, it was found that image sharpening improved model accuracy. This implies that access to better quality imagery, such as that available through paid for services, should considerably improve model precision and training times.

The results of this research have several important implications. First, the study used satellite data to predict the number and types of ships in three important cities in the Gulf of California. The resulting analysis can contribute to the region's shipping fleet composition, level of activity and ultimately their carbon inventory by adding the emissions produced by the small boat fleet. Further, through this approach, it is also possible to assign emissions into regions supporting the development of policies that can mitigate local GHG and air pollution. In addition, the transfer learning algorithm can be pre-trained in advance and immediately applied to any sea area worldwide. This will provide a potential method to increase efficiency for scientists and engineers worldwide who need to estimate local maritime emissions. In addition, the model can quickly and accurately identify the boat's length and classify them, allowing researchers to allocate more time to the vessels they need concentrate on, not just small boats. Finally, all of the above benefits can be exploited in under served areas with a shortage of infrastructure and resources.

This work is the first step to building emission inventories through image recognition, and it has some limitations. The study considered the ship as a single detection object. It did not evaluate whether the model can improve the accuracy of identifying ships in the case of multiple detection objects. For instance, BoatNet was not trained to detect docks to improve the metrics of detecting boats. By down-sampling the image to 416 pixels × 416 pixels, it is possible to mask some of the boats at the edges of the photograph.
Furthermore, deep learning models train faster on small images~\cite{tan2021efficientnetv2}. A larger input image requires the neural network to learn from four times as many pixels, increasing the architecture's training time. In this work, a considerable proportion of the images in the dataset were large images of 4800 pixels x 2908 pixels. Thus, BoatNet was set to learn from resized small images measuring 416 pixels x 416 pixels. Due to the low data quality of the selected regions, the images are less suitable as training datasets. However, using datasets from other regions or higher quality open-source imagery may result in inaccurate coverage of all types of ships in the region. When focusing on the small boat categorisation and the data used, understanding the implications of different environments (e.g. water or land) on object classification accuracy through the AI fairness principle deserves further study. From this point of view, large-scale collection of data sources in the real physical world would be costly and time-consuming. That said, it is possible that reinforcement learning, or building simulations in the virtual world, could reduce the negative impact of the environment on object recognition and thus improve its categorisation precision. Of all these limitations, model detection still achieves excellent performance in detecting and classifying small boats.

It is important to remember that BoatNet currently only detects and classifies certain types of small boats. Therefore, to estimate fuel consumption and emissions, it is necessary to couple it with small boat behaviour datasets~\cite{ferrer2021mexican}, typical machinery, fuel characteristics, and emission factors unique to this maritime segment~\cite{inecc2020inventario}. 

Finally, this work has demonstrated that deep learning models have the potential to identify small boats in extreme environments at performance levels that provide practical value. With further analysis and small boat data sources, these methods may eventually allow for the rapid assessment of shipping carbon inventories.