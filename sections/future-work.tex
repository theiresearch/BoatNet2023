Although the location and size of small boats have been successfully identified, and the boats successfully but roughly classified into domestic recreational and fishing categories, more work is still needed to refine and improve them. In the future,

\begin{enumerate}
    
    \item in determining whether or not the boats are white, I created a function with RGB as the independent variable. I required the output of this function to be greater than a particular value (threshold). However, adjusting the parameters of this function can, in a certain sense, only be based on experience. So, does this threshold accurately identify the boat as being white in colour?
    
    \item if it is possible to roughly determine the type of small boat based on colour alone, is it possible that a dark coloured fishing boat is not detected on dark coloured seas? Would the number of fishing boats, in fact, then be greater?
    
    \item the satellite images capture only a moment in time during the year, are the fishing vessels operating slightly further out to sea and not captured by the satellite due to the nature of the fishing vessels?


    \item can data providers provide scholars with clear and close satellite imagery for free? Data is one of the triads of artificial intelligence and also the most overlooked factor. In fact, AI algorithms are highly dependent on data. It is as if the algorithms are always hungry, and all require a constant stream of data to feed this hunger. For object detection algorithms, poor image detail representation means less high quality data. For the algorithm, less data often reuslts in poor output results. Therefore, both the data used in building the model and that used for detection should be maintained at the highest level. Otherwise, training AI models loses its relative meaning.
    
    \item will Google Earth Pro provide a deep integration tool about the zoom scale in the future? Google Earth Pro provides many images with detail, however, it does not offer similar integrated tools for zooming rulers as possible in Google Maps. If such a scaling environment were available, it would result in every photo taken on Google Earth Pro containing its scaling ratio. This means that we do not need to define a fixed zoom scale. It would be easy to know the object's length in each screenshot, regardless of whether or not the eye altitude is 200 metres. This leads to the fact that when it is necessary to detect some giant cargo ships or cruise ships, we can reduce the scale to obtain an overall picture of the giant ship. When some tiny ships (e.g., a small ship under 5 metres in length) need to be inspected, we can zoom in to get their overall shape.
    
    \item will small boats have a more precise classification in the future? Using colour to distinguish whether a boat is a recreational boat or a fishing boat may not be a solution that is acceptable to everyone. However, to solve this problem with a purely deep learning approach, it is necessary to train the data with all the types of boats that require detection. However, undertaking the data classification and training in a realistic environment, i.e. Google Earth Pro, would be labour-intensive and costly. Furthermore, when looking for the classification category and the data under this category, is it possible to guarantee that the object's environment is the same as its environment under other categories due to the AI fairness principle? For example, is it possible to guarantee that the proportion of large cruise ships appearing on the beach is the same as that of small recreational boats appearing on the shore? The reason for this is that we do not want all large cruise ships to be offshore and all small recreational boats to be on the beach. If this is the case, the deep neural network does not need to know the contours or characteristics of the different boats. It just needs to analyse the background of boats (the colour of the sea level, colour of the beach, etc.) to predict whether a large cruise ship is likely to be a large cruise ship in the first place. One of the potential solutions to this problem is using computer simulation techniques to create CAD models of various ships. These CAD ships are then placed evenly into a variety of different background environments. Finally, these data are trained with deep neural networks, and then the trained models are placed into real-life situations to detect some real ships.
    
    \item *will it be possible in the future to quickly scale up the algorithm to other regions? Additionally, will it be possible to analyse real-time satellite video in the future? Again, the key to expanding the algorithm to other regions and analysing real-time satellite video is to have fast and timely access to the data in the region. Frankly, it is whether data providers can supply information in the identified region. With real-time satellite video, it will be possible to obtain a better picture of maritime traffic in any area to monitor and control the carbon emissions of shipping in that area.
    
    \item is it possible to add more recognition objects, such as harbour and sea ripples, to prevent the recognition of harbour or sea ripples as small boats? The purpose of this is that once the algorithm can successfully identify these characteristics, they will not be identified as small boats.
    
    \item can the timeframes be improved? It is a question of optimisation, a trade-off question of cloud storage, a question of the algorithm's speed, and a question of the accuracy of the result. Besides, can the algorithm distinguish the same boat?
    
    \item Can we get a better idea of the type of engine on the boat? The connection of the ship is rightly allocated but it is still possible that difficulty may arise in identifying what type of engine or power is installed onboard. For this, a better understanding between ship size and type and the levels of power needed is required. This suggestion opens a new door of research.
\end{enumerate}
 
